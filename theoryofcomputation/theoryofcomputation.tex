\input{../mac/head-default.tex}
\input{../mac/maths.tex}

\Bdc{Notes on the Theory of Computation}

\section{Automata and Formal Languages}

An {\bf alphabet} is a finite set \(\varSigma\), and a {\bf word over the
alphabet \(\varSigma\)} is a finite sequence or {\bf string} of the elements of
\(\varSigma\). If a word \(w\) is the sequence \((w_0, \ldots, w_n)\) for some
\(n \in \setnat\), we may write the word as \(w_0 \cdots w_n\). The empty word
is denoted by \(\varepsilon\). The set of all words over \(\varSigma\) is
\(\varSigma^*\)\footnote{\(^*\) is the unary operator of Kleene star, defined as
\(A^* = \set{a_0 \cdots a_n : n \in \setnat \land \forall i \in \Nln[n + 1] (a_i
\in A)}\).}. A {\bf formal language over the alphabet \(\varSigma\)} is a subset
of \(\varSigma^*\).

An {\bf automaton} is an ordered sequence that {\bf accepts} some words over an
alphabet. The set of words an automaton accepts forms a language, which is
unique, in which case we say the automaton {\bf recognises} the language. Given
an automaton \(M\), we may speak of the unique language recognised by \(M\) as
the {\bf language of the automaton \(M\)} and denote it by \(L(M)\). A machine
may accept no string, in which case the language thereof is \(\nset\).

\subsection{Finite-State Automata and Regular Languages}

\Bdf
    A {\bf finite-state automaton} is an ordered quintuple \((\varSigma, S,
    \delta, s_0, F)\) wherein
    \begin{enumerate}
        \item \(\varSigma\) is an alphabet,
        \item \(S\) is a finite set of {\bf states},
        \item \(\map{\delta}{S \times \varSigma}{S}\) is the {\bf transition
        function},
        \item \(s_0 \in S\) is the {\bf initial state}, and
        \item \(F \subseteq S\) is the set of {\bf final states} or {\bf accept
        states}.
    \end{enumerate}
\Edf

Let \(M = (\varSigma, S, \delta, s_0, F)\) be a finite-state automaton and let
\(w = w_0 \cdots w_n\) wherein \(n \in \setnat\) be a word over \(\varSigma\).
Then \(M\) accepts \(w\) if a finite sequence of states \((r_0, \ldots, r_n)\)
in \(S\) exists such that
\begin{enumerate}
    \item \(r_0 = q_0\),
    \item \(\delta(r_i, w_i) = r_{i + 1}\) for \(i = 0, \ldots, n - 1\), and
    \item \(r_n \in F\).
\end{enumerate}

\Bdf
    Let \(\varSigma\) be an alphabet, and let \(a \in \varSigma\). Then some \(R
    \subseteq \varSigma^*\) is a {\bf regular language} if
    \begin{enumerate}
        \item \(R = \nset\),
        \item \(R = \set{\varepsilon}\),
        \item \(R = \set{a}\),
        \item \(R = R_1 \cup R_2\) wherein \(R_1\) and \(R_2\) are regular
        languages over \(\varSigma\),
        \item \(R = R_1 R_2\)\footnote{\(R_1 R_2\) is the concatenation of
        \(R_1\) and \(R_2\), defined as \(R_1 R_2 = \set{x y : x \in R_1 \land y
        \in R_2}\)} wherein \(R_1\) and \(R_2\) are regular languages over
        \(\varSigma\), or
        \item \(R = R_0^*\) wherein \(R_0\) is a regular language over
        \(\varSigma\).
    \end{enumerate}
\Edf

An expressed used to characterise a regular language is a {\bf regular
expression}.

\Edc