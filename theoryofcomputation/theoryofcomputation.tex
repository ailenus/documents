\documentclass[12pt]{article}

\makeatletter
\def\@maketitle {
    \newpage
    \null
    \vskip 2em
    \begin{center}
        \let \footnote \thanks
        {
            \Large\bfseries \MakeUppercase{\@title} \par
        }
        \vskip 1.5em
        {
            \large \lineskip .5em \begin{tabular}[t]{c}
                \MakeUppercase{\@author}
            \end{tabular} \par
        }
        \vskip 1.5em
        {
            \large\scshape \@date
        }
    \end{center}
    \par
    \vskip 1.5em
}
\makeatother

%%%%%%%%%%%%%%%%%%%%%%%%%%%%%%%%%%%%%%%%%%%%%%%%%%%%%%%%%%%%%%%%%%%%%%%%%%%%%%%
%% Packages
%%%%%%%%%%%%%%%%%%%%%%%%%%%%%%%%%%%%%%%%%%%%%%%%%%%%%%%%%%%%%%%%%%%%%%%%%%%%%%%

%% basic
\usepackage[a4paper,margin=1in]{geometry}
\usepackage{fontspec}
\usepackage{polyglossia}

%% formatting
\usepackage{setspace}
\usepackage{enumitem}
\usepackage{tocloft}
\usepackage{titlesec}
\usepackage{fancyhdr}

%% maths
\usepackage{amsmath}
\usepackage{amsthm}
\usepackage{mathtools}
\usepackage[math-style=ISO,bold-style=ISO]{unicode-math}

%% utilities
\usepackage{array}
\usepackage{booktabs}
\usepackage[colorlinks=true]{hyperref}

%%%%%%%%%%%%%%%%%%%%%%%%%%%%%%%%%%%%%%%%%%%%%%%%%%%%%%%%%%%%%%%%%%%%%%%%%%%%%%%
%% Settings
%%%%%%%%%%%%%%%%%%%%%%%%%%%%%%%%%%%%%%%%%%%%%%%%%%%%%%%%%%%%%%%%%%%%%%%%%%%%%%%

%% basic
\setmainlanguage[variant=british]{english}
\PolyglossiaSetup{english}{indentfirst=true}
\setmainfont{Libertinus Serif}
\setsansfont{Libertinus Sans}
\setmonofont{Libertinus Mono}
\setmathfont[AutoFakeBold]{Libertinus Math}

%% formatting
\doublespacing
\setlist{noitemsep}
\setlist[1]{labelindent=\parindent}
\setlist[1]{listparindent=\parindent}
\setlist[enumerate,1]{label=(\alph*)}
\setlist[enumerate,2]{label=(\roman*)}
\setlength{\headheight}{15.2pt}
\setlength{\cftsecindent}{0em}
\setlength{\cftsubsecindent}{0em}
\setlength{\cftsubsubsecindent}{0em}
\renewcommand*{\cfttoctitlefont}{\large\scshape}
\renewcommand*{\cftsecfont}{\normalfont}
\renewcommand*{\cftsecpagefont}{\normalfont}
\renewcommand*{\cftsecleader}{\cftdotfill{\cftdotsep}}
\titleformat{\section}{
    \normalfont\Large\scshape\filcenter
}{}{1em}{}
\titleformat{\subsection}{
    \normalfont\large\scshape
}{\thesubsection}{1em}{}
\titleformat{\subsubsection}{
    \normalfont\normalsize\scshape
}{\thesubsubsection}{1em}{}
\setcounter{footnote}{1}

%% maths
\newtheoremstyle{dfn}
    {\topsep}
    {\topsep}
    {\upshape}
    {0pt}
    {\scshape}
    {.}
    {5pt plus 1pt minus 1pt}
    {}
\newtheoremstyle{thm}
    {\topsep}
    {\topsep}
    {\slshape}
    {0pt}
    {\bfseries}
    {.}
    {5pt plus 1pt minus 1pt}
    {}
\newtheoremstyle{xrc}
    {\topsep}
    {\topsep}
    {\upshape}
    {0pt}
    {\bfseries\scshape}
    {.}
    {5pt plus 1pt minus 1pt}
    {}
\newtheoremstyle{sol}
    {\topsep}
    {\topsep}
    {\upshape}
    {0pt}
    {\itshape}
    {.}
    {5pt plus 1pt minus 1pt}
    {}
\theoremstyle{dfn}
\newtheorem{dfn}{Definition}
\theoremstyle{thm}
\newtheorem{thm}{Theorem}
\newtheorem{lem}{Lemma}
\newtheorem{cor}{Corollary}
\theoremstyle{xrc}
\newtheorem{xrc}{Exercise}
\theoremstyle{sol}
\newtheorem*{XxmpX}{Solution}
\newenvironment{solution}{
    \renewcommand{\qedsymbol}{\(\mdlgwhtlozenge\)}
    \pushQED{\qed}\begin{XxmpX}
}{\popQED\end{XxmpX}}

%% aliases
\newcommand*{\Bdc}[2][\today]{
    \title{#2}
    \author{Yannan Mao}
    \date{#1}
    \lhead{\scshape #2}
    \rhead{\scshape Yannan Mao}
    \begin{document}
    \maketitle
    \tableofcontents
    \thispagestyle{empty}
    \clearpage
    \pagestyle{fancy}
    \setcounter{page}{1}
}
\newcommand*{\Edc}{\end{document}}
\newcommand*{\Bdf}{\begin{dfn}}
\newcommand*{\Edf}{\end{dfn}}
\newcommand*{\Bth}{\begin{thm}}
\newcommand*{\Eth}{\end{thm}}
\newcommand*{\Blm}{\begin{lem}}
\newcommand*{\Elm}{\end{lem}}
\newcommand*{\Bcr}{\begin{cor}}
\newcommand*{\Ecr}{\end{cor}}
\newcommand*{\Bxr}{\begin{xrc}}
\newcommand*{\Exr}{\end{xrc}}
\newcommand*{\Bpr}{\begin{proof}}
\newcommand*{\Epr}{\end{proof}}
\newcommand*{\Bsl}{\begin{solution}}
\newcommand*{\Esl}{\end{solution}}
\DeclareMathOperator{\dom}{dom}             % domain
\DeclareMathOperator{\im}{im}               % image
\DeclareMathOperator*{\argmax}{arg\,max}    % arg max
\DeclareMathOperator*{\argmin}{arg\,min}    % arg min

\DeclarePairedDelimiter{\abs}{\lvert}{\rvert}               % absolute value, cardinality
\DeclarePairedDelimiter{\norm}{\lVert}{\rVert}              % norm
\DeclarePairedDelimiterX{\inn}[2]{\langle}{\rangle}{#1,#2}  % inner product
\DeclarePairedDelimiter{\set}{\{}{\}}                       % set

\newcommand*{\ee}{\ensuremath{\mathrm{e}}}              % e
\newcommand*{\ii}{\ensuremath{\mathrm{i}}}              % i
\newcommand*{\diff}{\ensuremath{\mathop{}\!\mathrm{d}}} % differential

\newcommand*{\pow}{\ensuremath{\mathcal{P}}}            % power set
\newcommand*{\nset}{\ensuremath{\emptyset}}             % null set
\newcommand*{\setnat}{\ensuremath{\mathbb{N}}}          % set of natural numbers
\newcommand*{\setint}{\ensuremath{\mathbb{Z}}}          % set of integers
\newcommand*{\setrat}{\ensuremath{\mathbb{Q}}}          % set of rational numbers
\newcommand*{\setreal}{\ensuremath{\mathbb{R}}}         % set of real numbers
\newcommand*{\setcomp}{\ensuremath{\mathbb{C}}}         % set of complex numbers
\newcommand*{\posint}{\ensuremath{\setint_{>0}}}        % set of positive integers
\newcommand*{\posreal}{\ensuremath{\setreal_{>0}}}      % set of positive real numbers
\newcommand*{\nonneg}{\ensuremath{\setreal_{\ge 0}}}    % set of nonnegative numbers
\newcommand*{\Nln}[1][n]{\ensuremath{\setnat_{<#1}}}    % set of natural numbers less than n

\newcommand*{\vct}[1]{\ensuremath{\symbf{#1}}}      % vector
\newcommand*{\map}[3]{\ensuremath{#1\colon#2\to#3}} % map
\newcommand*{\grad}{\ensuremath{\nabla}}            % gradient

\newcommand*{\dis}{\ensuremath{\displaystyle}}


\Bdc{Notes on the Theory of Computation}

\section{Automata and Formal Languages}

An {\bf alphabet} is a finite set \(\varSigma\), and a {\bf word over the
alphabet \(\varSigma\)} is a finite sequence or {\bf string} of the elements of
\(\varSigma\). If a word \(w\) is the sequence \((w_0, \ldots, w_n)\) for some
\(n \in \setnat\), we may write the word as \(w_0 \cdots w_n\). The empty word
is denoted by \(\varepsilon\). The set of all words over \(\varSigma\) is
\(\varSigma^*\)\footnote{\(^*\) is the unary operator of Kleene star, defined as
\(A^* = \set{a_0 \cdots a_n : n \in \setnat \land \forall i \in \Nln[n + 1] (a_i
\in A)}\).}. A {\bf formal language over the alphabet \(\varSigma\)} is a subset
of \(\varSigma^*\).

An {\bf automaton} is an ordered sequence that {\bf accepts} some words over an
alphabet. The set of words an automaton accepts forms a language, which is
unique, in which case we say the automaton {\bf recognises} the language. Given
an automaton \(M\), we may speak of the unique language recognised by \(M\) as
the {\bf language of the automaton \(M\)} and denote it by \(L(M)\). A machine
may accept no string, in which case the language thereof is \(\nset\).

\subsection{Finite-State Automata and Regular Languages}

\Bdf
    A {\bf finite-state automaton} is an ordered quintuple \((\varSigma, S,
    \delta, s_0, F)\) wherein
    \begin{enumerate}
        \item \(\varSigma\) is an alphabet,
        \item \(S\) is a finite set of {\bf states},
        \item \(\map{\delta}{S \times \varSigma}{S}\) is the {\bf transition
        function},
        \item \(s_0 \in S\) is the {\bf initial state}, and
        \item \(F \subseteq S\) is the set of {\bf final states} or {\bf accept
        states}.
    \end{enumerate}
\Edf

Let \(M = (\varSigma, S, \delta, s_0, F)\) be a finite-state automaton and let
\(w = w_0 \cdots w_n\) wherein \(n \in \setnat\) be a word over \(\varSigma\).
Then \(M\) accepts \(w\) if a finite sequence of states \((r_0, \ldots, r_n)\)
in \(S\) exists such that
\begin{enumerate}
    \item \(r_0 = q_0\),
    \item \(\delta(r_i, w_i) = r_{i + 1}\) for \(i = 0, \ldots, n - 1\), and
    \item \(r_n \in F\).
\end{enumerate}

\Bdf
    Let \(\varSigma\) be an alphabet, and let \(a \in \varSigma\). Then some \(R
    \subseteq \varSigma^*\) is a {\bf regular language} if
    \begin{enumerate}
        \item \(R = \nset\),
        \item \(R = \set{\varepsilon}\),
        \item \(R = \set{a}\),
        \item \(R = R_1 \cup R_2\) wherein \(R_1\) and \(R_2\) are regular
        languages over \(\varSigma\),
        \item \(R = R_1 R_2\)\footnote{\(R_1 R_2\) is the concatenation of
        \(R_1\) and \(R_2\), defined as \(R_1 R_2 = \set{x y : x \in R_1 \land y
        \in R_2}\)} wherein \(R_1\) and \(R_2\) are regular languages over
        \(\varSigma\), or
        \item \(R = R_0^*\) wherein \(R_0\) is a regular language over
        \(\varSigma\).
    \end{enumerate}
\Edf

An expressed used to characterise a regular language is a {\bf regular
expression}.

\Edc